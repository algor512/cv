\documentclass{resume}
\usepackage{hyperref}
\usepackage[T2A]{fontenc}
\usepackage[utf8]{inputenc}
% \usepackage[russian]{babel}
\usepackage{lipsum}
\usepackage{xcolor}
\usepackage{hyperref}
\usepackage[left=0.75in,top=0.6in,right=0.75in,bottom=0.6in]{geometry}

\hypersetup{
    colorlinks=false,
    urlbordercolor=blue,
    pdfborderstyle={/S/U/W 1}
}

\name{Alexey Gorelov}
\address{+79262643463 \\ \href{mailto:algor512@gmail.com}{algor512@gmail.com}}

\begin{document}
\vspace{1em}

In 2015 I got a specialist's degree in Applied Mathematics and Informatics at Lomonosov Moscow State
University. There I studied data analysis and machine learning. My thesis work focused on developing
a probabilistic model for message passing delays between nodes of a computing cluster.

After the graduation, I decided to gain practical experience in the field of data analysis so I got a
job at Mail.Ru Group. There I worked on revenue predictions, planning and analysis of experiments,
and so on. Also, I gained strong programming experience there (Python, Java, Hadoop).

But after several years I began to feel that my work has turned into a routine and that I really
missed the beauty of mathematics. In 2019 I entered a Master's degree program in Mathematics at
Higher School of Economics, and I got the Master's degree in 2021. In the same year I started my PhD
program at Steklov Mathematical Institute of RAS.

I like the combining of geometrical beauty and combinatorial technics so the fields of my interest
are piecewise linear topology, topological graph theory, and topological combinatorics. My 1st year
course work was on the characterization of collapsible polyhedra in terms of free deformation
contractibility, while my thesis work was about the problem of an existence of a lifting to an
embedding of a map between graphs.
\vspace{1em}

\begin{rSection}{Education}
\begin{rSubsection}{PhD in Mathematics}{October 2021 --- Present}{}
  Steklov Mathematical Institute of Russian Academy of Sciences\vspace{0.5em}

  Supervisor: Sergey Melikhov
\end{rSubsection}

\begin{rSubsection}{M.Sc. in Mathematics (with excellence)}{September 2019 --- June 2021}{}
  Faculty of Mathematics, \\
  National Research University Higher School of Economics \vspace{0.5em}

  Supervisor: Sergey Melikhov \\
  Thesis theme: Lifting maps between graphs to embeddings \\
  GPA: 9.33 out of 10
\end{rSubsection}

\begin{rSubsection}{Specialist's degree in Applied Mathematics and Informatics}{September 2010 --- June 2015}{}
  Faculty of Computational Mathematics and Cybernetics, \\
  Lomonosov Moscow State University \vspace{0.5em}

  Supervisor: Archil Maysuradze \\
  Thesis theme: Анализ задержек в коммуникационной среде вычислительного кластера [Message passing
  delays analyses in the communication network of a computing cluster] \\
  GPA: 4.71 out of 5
\end{rSubsection}
\end{rSection}

\begin{rSection}{Additional education}
  \begin{rSubsection}{Autumn school ``Toric topology and combinatorics''}{1---5 November 2021}{} 
    Sirius Mathematics Center, Sochi, Russia
  \end{rSubsection}
\end{rSection}

\begin{rSection}{Publications}
  \begin{rSubsection}{Geometry of collapsing and free deformation retraction}{2021}{Alexey Gorelov}{}
    \href{https://arxiv.org/abs/2103.16464}{arXiv:2103.16464 [math.GT]}
  \end{rSubsection}

  \begin{rSubsection}{Информационная модель для снятия многозначности морфемного разбора в татарском
      языке [Information model for morphological disambiguation in the Tatar
      language]}{2018}{Горелов А.А., Майсурадзе A.И. [Gorelov A., Maysuradze A.]}
    Дискретные модели в теории управляющих систем: Х Международная конференция, Москва и Подмосковье,
    23-25 мая 2018 : Труды, том 1, с. 104-106
  \end{rSubsection}

  \begin{rSubsection}{Анализ структуры задержек передачи информации в вычислительном кластере
      [Delay
      structure mining in a computing cluster]}{2015}{А.А. Горелов, А.И. Майсурадзе, А.Н. Сальников [Gorelov A., Maysuradze A., Salnikov A.]}{}
    Proceedings of the 1st Russian Conference on Supercomputing - Supercomputing Days 2015
  \end{rSubsection}
\end{rSection}

\begin{rSection}{Conference presentations}
  \begin{rSubsection}{Geometry of collapsing and free deformation retraction}{2021}{Poster presentation}
    Young Topologists Meeting 2021, Stockholm, Sweden (online)
  \end{rSubsection}

  \begin{rSubsection}{Geometry of collapsing and free deformation retraction}{2021}{Poster presentation}
    Graduate Student Topology and Geometry Conference 2021, Indiana University, USA (online)
  \end{rSubsection}

  \begin{rSubsection}{Анализ структуры задержек передачи информации в вычислительном кластере [Delay
      structure mining in a computing cluster]}{2015}{Poster presentation}
    Russian Supercomputing Days 2015, Moscow, Russia
  \end{rSubsection}
  
  \begin{rSubsection}{Восстановление зерновых вершин графа, полученного поиском в
      ширину [Reconstructing the seed vertices of a graph obtained by the breadth-first search
      algorithm]}{2013}{Oral presentation}
    International student, postgraduate and young scientist conference ``Lomonosov-2013'', Moscow, Russia
  \end{rSubsection}
\end{rSection}

\begin{rSection}{Teaching experience}
  \begin{rSubsection}{Visiting scholar}{January 2021 --- April 2021}{}
    Conducting seminars for the course ``Linear Algebra'' for 1st year undergraduate students, \\
    Graduate School of Business, \\
    National Research University Higher School of Economics
  \end{rSubsection}

  \begin{rSubsection}{Teaching assistant}{January 2020 --- May 2020}{}
    Course ``Mathematical analysis'' for 1st year undergraduate students, \\
    Faculty of Mathematics, \\
    National Research University Higher School of Economics
  \end{rSubsection}

  \begin{rSubsection}{Visiting scholar}{September 2019 --- December 2019}{}
    Conducting seminars for the course ``Algebra and Geometry'' for 1st year undergraduate students, \\
    Graduate School of Business, \\
    National Research University Higher School of Economics
  \end{rSubsection}
\end{rSection}

\begin{rSection}{Scholarships and Grants}
  \begin{rSubsection}{Special scholarship for HSE master's students}{2020 --- 2021}{}
  \end{rSubsection}

  \begin{rSubsection}{Russian Foundation for Basic Research grant №15--07--09214}{2017}{(one of the participants)}
  \end{rSubsection}
\end{rSection}

\begin{rSection}{Seminar reports}
  \begin{rSubsection}{Поднятие погружений до вложений в коразмерности один [Lifting immersions to embeddings in codimension one]}{23 February 2022}{}
    Geometric Topology Seminar, Steklov Mathematical Institute of RAS, Moscow, Russia
    \vspace{0.5em}

    \href{http://www.mathnet.ru/php/seminars.phtml?option_lang=eng&presentid=34048}{Slides and recordings (in Russian)}.
  \end{rSubsection}

  \begin{rSubsection}{Поднятие отображений между графами во вложение [Lifting maps between graphs to embeddings]}{28 May 2021}{}
    Geometric Topology Seminar, Steklov Mathematical Institute of RAS, Moscow, Russia
    \vspace{0.5em}

    \href{http://www.mathnet.ru/php/seminars.phtml?option_lang=eng&presentid=30731}{Recordings (in Russian)}.
  \end{rSubsection}

  \begin{rSubsection}{Аппроксимация вложениями отображений графов в плоскость [Approximation by embeddings of maps of graphs into the plane]}{28 April, 2 May 2021}{}
    Geometric Topology Seminar, Steklov Mathematical Institute of RAS, Moscow, Russia
    \vspace{0.5em}

    Recordings (in Russian) of the
    \href{http://www.mathnet.ru/php/seminars.phtml?option_lang=eng&presentid=30406}{first part} and
    the \href{http://www.mathnet.ru/php/seminars.phtml?option_lang=eng&presentid=30482}{second part}.
  \end{rSubsection}

  \begin{rSubsection}{Некоторые характеризации CAT(0) кубических комплексов [Some characterizations
      of CAT(0) cubical complexes]}{11, 18 December 2020}{}
    Geometric Topology Seminar, Steklov Mathematical Institute of RAS, Moscow, Russia
    \vspace{0.5em}

    Slides and recordings (in Russian) of the
    \href{http://www.mathnet.ru/php/seminars.phtml?option_lang=eng&presentid=28954}{first part} and
    the \href{http://www.mathnet.ru/php/seminars.phtml?option_lang=eng&presentid=29250}{second part}.
  \end{rSubsection}

  \begin{rSubsection}{The Kister-Mazur theorem}{8 December 2020}{}
    Seminar on smooth, PL- and topological manifolds, Faculty of Mathematics, Moscow, Russia
    \vspace{0.5em}

    \href{https://drive.google.com/file/d/1WE5F7Bgqm7x6pB6irsLrthQtfk3Psl7e/view}{Slides}.
  \end{rSubsection}

  \begin{rSubsection}{Сдавливание и свободная деформационная ретракция [Collapsing and free
      deformation retraction]}{5 June 2020}{}
    Seminar of International Laboratory of Algebraic Topology and Its Applications, Faculty of
    Computer Sciences of HSE, Moscow, Russia
  \end{rSubsection}

  \begin{rSubsection}{Сдавливание и свободная деформационная ретракция [Collapsing and free deformation retraction]}{13 May 2020}{}
    Geometric Topology Seminar, Steklov Mathematical Institute of RAS, Moscow, Russia
    \vspace{0.5em}

    \href{http://www.mathnet.ru/php/seminars.phtml?option_lang=eng&presentid=27125}{Slides and recordings (in Russian)}.
  \end{rSubsection}

  \begin{rSubsection}{Медианные пространства и выпуклые структуры [Median spaces and convex structures]}{11 December 2019}{}
    Geometric Topology Seminar, Steklov Mathematical Institute of RAS, Moscow, Russia
    \vspace{0.5em}

    \href{http://www.mathnet.ru/php/seminars.phtml?option_lang=eng&presentid=26027}{Slides and recordings (in Russian)}.
  \end{rSubsection}
\end{rSection}

\begin{rSection}{Student olympiads participation}
  \begin{rSubsection}{HSE Olympiad competition for students and graduates, track ``Mathematics''}{2019}{}
    II degree diploma
  \end{rSubsection}
\end{rSection}

\begin{rSection}{Work experience}
  \begin{rSubsection}{Mail.ru Group}{September 2015 --- October 2019}{}
    Data analyst and data scientist in Mail.ru Search
  \end{rSubsection}
\end{rSection}
\pagebreak

\begin{rSection}{Other education}
  \begin{rSubsection}{Creative Writing School}{October 2015 --- December 2015}{} 
    Poetry workshop, supervised by Dmitry Bykov
  \end{rSubsection}
\end{rSection}

\begin{rSection}{Professional Skills}
  \begin{rSubsection}{Programming languages}{}{}
    Working knowledge: Python, Java, bash, C \\
    Basic knowledge: C++, Matlab, R, Go, Kotlin, C\#
  \end{rSubsection}

  \begin{rSubsection}{Software and technologies}{}{}
    GNU/Linux (I've been working with GNU/Linux since 2008), Python scientific stack (numpy, scipy, sympy, matplotlib), Hadoop, Matlab, VCS
  \end{rSubsection}

  \begin{rSubsection}{Languages}{}{}
    Russian: native speaker \\
    English: upper intermediate
  \end{rSubsection}
\end{rSection}
\end{document}
